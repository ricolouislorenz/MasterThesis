%!TEX root = thesis.tex
%---------------------------------------------------------------------------
% Frontpage
%---------------------------------------------------------------------------

% Die Richtline zum Aufbau des Deckblatts von Bachelor- und Masterarbeiten
% findet sich hier:
% @see: http://www.uni-luebeck.de/fileadmin/uzl_ssc/PDF-Dateien/Richtlinie-Deckblatt-MINT-Abschlussarbeit-2012-10-18.pdf

\newcommand{\titlepageskip}{\vskip 20pt}

% @see: http://tex.stackexchange.com/questions/31705/different-margins-for-title-page
\newgeometry{top=1in,bottom=1in,right=1in,left=1.2in}
\begin{titlepage}

\title{Deutscher Titel der Bachelor-/Masterarbeit}
\author{Max Mustermann}

{\Large
	% 1. Offizielles Logo des Instituts, an dem die Arbeit angesiedelt ist. (Das offizielle Logo
	% enthält das Siegel der Universität zusammen mit dem Text "Universität zu Lübeck"
	% und darunter den Namen des Instituts.) Dieses Logo ist bei den Instituten zu
	% bekommen. Das Logo muss oben links platziert werden.
	\includegraphics[width=80mm]{Logo_Inst_Telematik_cropped}
	\vskip 44pt

	% 2. Optional: Noch einmal Name des Instituts und Angabe der Direktorin oder des
	% Direktors des Instituts.

	% 3. Titel der Arbeit in deutscher Sprache und ebenfalls in englischer Sprache. Dabei soll
	% die Sprache, in der die Arbeit verfasst wurde, als erste angeführt werden; die andere
	% Sprache kann weniger prominent dargestellt werden.
	% Auch bei englischsprachigen Studiengängen sollen die Titelblätter auf Deutsch sein.
	\textbf{\LARGE Deutscher Titel der Bachelor-/Masterarbeit, welcher über mehrere Zeilen gehen kann}
	\textbf{\LARGE Englischer Titel der Bachelor-/Masterarbeit, welcher über mehrere Zeilen gehen kann}

	\titlepageskip
	% 4. Der Text "Bachelorarbeit" oder "Masterarbeit" (nicht "Bachelor-Arbeit" oder "Master-Arbeit").
	%\textbf{Bachelorarbeit}
	\textbf{Masterarbeit}

	\titlepageskip
	%5. Der Text "im Rahmen des Studiengangs"
	im Rahmen des Studiengangs\\
	%6. Der ausgeschriebene Name des Studiengangs (also beispielsweise "Informatik"
	%oder "Molecular Life Science", hingegen nicht "Bioinformatik" oder "MLS")
	\textbf{Informatik}\\
	%7. Der Text "der Universität zu Lübeck"
	der Universit"at zu L"ubeck

	\titlepageskip
	%8. Der Text "Vorgelegt von" und der Name der Studentin oder des Studenten
	vorgelegt von\\
	\textbf{Max Mustermann}

	\titlepageskip
	%9. Der Text "Ausgegeben und betreut von"
	ausgegeben und betreut von\\
	%10. Der Name der ersten Prüferin oder des ersten Prüfers. Dies ist immer gleichzeitig
	%die Betreuerin oder der Betreuer im Sinne der Prüfungsordnung.
	\textbf{Prof.~Dr.~Stefan~Fischer}

	% Diesen Teil entfernen, wenn die Arbeit KEINEN Unterstützer hatte
	\titlepageskip
	{
		%11. Optional der Text "Mit Unterstützung von" und der Name von weiteren Personen,
		%die die Betreuung besonders unterstützt haben. Beispielsweise können dies
		%wissenschaftliche Mitarbeiter sein oder Mitarbeiter von Firmen, wenn die Arbeit
		%extern geschrieben wurde.
		mit Unterstützung von\\
		\textbf{Prof.~Dr.~X~Musterperson}\\
		\textbf{Dipl.-Inf.~Y~Mustermann}\\
	}

	% Diesen Teil entfernen, wenn die Arbeit NICHT in einer Firma entstanden ist
	\titlepageskip
	{
		%12. Optional ein Hinweis, dass die Arbeit zum Teil bei einer Firma entstanden ist wie
		%"Die Bachelorarbeit ist im Rahmen von Arbeiten bei Firma XY entstanden".
		Die Arbeit ist im Rahmen einer Tätigkeit bei der Firma Muster GmbH entstanden.
	}

	\vfill
	%13. Der Text "Lübeck, den" und das Abgabedatum.
	{
		Lübeck, den \today
	}

	% Diesen Teil entfernen, wenn "Im Focus das Leben" nicht drauf stehen soll
	%14. Optional der Text "Im Focus das Leben".
	{
		\titlepageskip
		Im Focus das Leben
	}
}
\end{titlepage}
\restoregeometry

\cleardoublepage

% Erklaerung
\newpage
\chapter*{Erkl"arung}

Ich versichere an Eides statt, die vorliegende Arbeit selbstst"andig und nur unter Benutzung
der angegebenen Hilfsmittel angefertigt zu haben.

\vspace*{3cm}
Lübeck, den \today

\thispagestyle{empty}
\cleardoublepage


% Kurzfassung und Abstract

\chapter*{Kurzfassung}

Diese Arbeit ganz kurz.

%
\vskip 3cm
%

\section*{\huge Abstract}

This thesis in short.

\thispagestyle{empty}
\cleardoublepage

% Aufgabenstellung

\chapter*{Aufgabenstellung}

Die Aufgabenstellung der Master-/Diplom-/Bachelor-/Studienarbeit