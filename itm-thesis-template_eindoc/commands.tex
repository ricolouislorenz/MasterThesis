%!TEX root = thesis.tex
% \KOMAoption{toc}{listof,bib,idx}

%------------------------------------------------------------------------------
%- PAKETE
%------------------------------------------------------------------------------
	%DIN A4
		\usepackage{a4}

	%Direkte Eingabe von Umlauten mit Angabe von Schriftsatz
	%in Kombination mit 'german' sind jetzt ä direkt erlaubt!
		\usepackage[T1]{fontenc}
		%\usepackage[latin1]{inputenc}
		\usepackage[utf8]{inputenc}

	%Sprache einstellen (Inhaltsverzeichnis, ...)
		\usepackage[ngerman]{babel} %american,italian,german,english


	%Euro Zeichen
		\usepackage{eurosym}

	% Vereinfachtes Eingeben von Leerschlägen hinter Shortcut-Commands
	% Beispiel: \newcommand{\DNA}{desoxyribose nucleid acid\xspace}
		\usepackage{xspace}

	%Sortierte Literaturverweise
		%\usepackage{cite}

	% Kommentare
		\usepackage{verbatim}

	%Verbessertes Beschriften mir div. Optionen
		\usepackage{caption}

	%Zusaetzliche Symbole und Schriften (ams: american mathematical soc)
		\usepackage{amssymb}
		%\usepackage{amstext		%\usepackage{amsfonts}
		%\usepackage{amsbsy}
		%\usepackage{amscd}
		%\usepackage{latexsym}

	%Text Companion fonts which provide many text symbols (such as baht, bullet, copyright, musicalnote, onequarter, section, and yen) in the TS1 encoding.
		\usepackage{textcomp}

	%Drehen von Text, Tabellen, Seiten
		%\usepackage{rotating}

	%Schöne professionelle Tabellen mit
		\usepackage{booktabs}
		\usepackage{array}
		\usepackage{tabularx}
		\usepackage{tabulary}
		\usepackage{multirow}

	%including graphics files, rotating parts of a page, and scaling parts of a page
		\usepackage{graphicx}

	%Nice drawing package
		%\usepackage{tikz}

	%besserer eps import: \eps import ERSETZEN durch \epsfig
		\usepackage{epsf}

	%Farbunterstützung (ausserhalb der Bilder)
		\usepackage{xcolor}
		\definecolor{gray}{gray}{.75}
		\definecolor{colorUzLgreen}{RGB}{0,75,90} 
		\definecolor{colorUzLred}{RGB}{228,32,50}
		\definecolor{colorUzLblue}{RGB}{0,106,163}
		\definecolor{colorUzLgray}{RGB}{208,208,208}

	% Ändern der Ränder für das Titelblatt
		\usepackage[]{geometry} % showframe

	%Postcript einbinden, wobei Text ersetzt werden kann
		%\usepackage{psfrag}

	%Für den Index
		\usepackage{makeidx}
		\makeindex %Muss vor begin{document}, sonst passiert nix

	%Erleichterungen fürs Deutsche inkl Silbentrennung
		%\usepackage{german}

	%Ensure minimal spacing of table cells (http://www.ctan.org/tex-archive/help/Catalogue/entries/cellspace.html)
		%\usepackage{cellspace}

	%Source code Listings
		\usepackage{listings}

	%Subfigures
		\usepackage{subfigure}

	% Ignore float add to list warning
	\usepackage{scrhack}

	%Automatically adds the bibliography and/or the index and/or the contents, etc., to the Table of Contents listing.
		%\usepackage[nottoc]{tocbibind} %,notlot,notlof

	%Add Bibliography Packages
		\usepackage[
  		style=ACM-Reference-Format,
  		backend=biber,
  		sortcites=true,
  		maxbibnames=99,
  		defernumbers=true
		]{biblatex}
		\usepackage{csquotes}

	%St Mary Road symbols for theoretical computer science.
		%\usepackage{stmaryrd}

	%URL Darstellung
		\usepackage{url}

	%PDF und Standard Latex Unterscheidung
		\usepackage{ifpdf}

	%Abkürzungsverzeichnis
		\usepackage[intoc]{nomencl}
		  \let\abbrev\nomenclature
		  \renewcommand{\nomname}{Abkürzungsverzeichnis}
		  \setlength{\nomlabelwidth}{.25\hsize}
		  \renewcommand{\nomlabel}[1]{#1 \dotfill}
		  \setlength{\nomitemsep}{-\parsep}
		  \makenomenclature

	  \newcommand{\abk}[2]{#1\abbrev{#1}{#2}}

	%With \usepackage{ulem}, you have the following new commands:
		%    * \uline{important} underlined text
		%    * \uuline{urgent} double-underlined text
		%    * \uwave{boat} wavy underline
		%    * \sout{wrong} line drawn through word
		%    * \xout{removed} marked over with //////.
		%    * {\em phasized\/} and \emph{asized} In LaTeX, by default, these are underlined; use \normalem or [normalem] to restore italics
		%    * \useunder{\uwave}{\bfseries}{\textbf} use wavy underline in place of bold face
		%Note that this package changes \em and \emph to be underline. To change this behavior back to normal, use the \normalem command, for example
		%\usepackage{ulem}
		%\normalem
		%\usepackage[normalem]{ulem}

	  %\newcommand{\markup}[1]{\uline{#1}}

	% package to customize the three basic lists (enumerate, itemize and description)
	% by means of a set of parameters, and to clone them to define new "logical" lists.
		\usepackage{enumitem}
		\setitemize{enumsep=-3pt}
		\setitemize{itemsep=-3pt}

	%Definitionen
		\usepackage{theorem}
		\newcounter{theorem}
		\newtheorem{definition}[theorem]{Definition}


	\usepackage[bookmarks=true,
					bookmarksopen=true,
   					% Lesezeichen ausgeklappt
					bookmarksnumbered=true,
					colorlinks=true,
				   	%Einfärbung von Links
					linkcolor=black
					% Linkfarbe: schwarz
					]
				    % Anzeige der Kapitelzahlen am Anfang der Namen der Lesezeichen
				   {hyperref}

	
	% Todos
		% Alle Todos und Liste aller Todos zeigen
		\usepackage[colorinlistoftodos]{todonotes}
		% Alle Todos und Liste aller Todos deaktivieren
		%\usepackage[disable]{todonotes}

	%Darstellung von Algorithmen
	\usepackage{algorithm}
	\usepackage{algorithmic}

	%Zitate
		\newcounter{quotectr}
		\newtheorem{myquote}[quotectr]{Zitat}

%------------------------------------------------------------------------------
%- Layout
%------------------------------------------------------------------------------

	%Tiefe des Inhaltsverzeichnisses und der Nummerierung der Kapitel
		\setcounter{secnumdepth}{2}
		\setcounter{tocdepth}{2}

	%Call this after each chapter to avoid headlines on empty pages
		\newcommand{\chapterfin}{\clearpage{\pagestyle{empty}\cleardoublepage}}
		\newcommand{\sectionfin}{\clearpage{\pagestyle{empty}\cleardoublepage}}

% % 	% Listings schoen machen
% % 		\renewcommand*\ttdefault{txtt}
 	%Umbenennung
 	\renewcommand{\lstlistingname}{Programmcode}

	\lstset{
		backgroundcolor=\color{white},
 		%basicstyle=\footnotesize\listingsfont,
		breaklines=true, 
 		captionpos=b, 
 		commentstyle=\color{colorUzLgreen}\itshape,
 		escapeinside={\%*}{*)}, 
 		frame=single,
 		keepspaces=true,
 		keywordstyle=\color{colorUzLblue},
 		numbers=left,
  		numberstyle=\tiny\color{gray}, 
 		rulecolor=\color{black},  
  		showspaces=false, 
 		showstringspaces=false, 
 		showtabs=false, 
 		stepnumber=1, 
 		tabsize=2,title=\lstname,
 		language=C++,
 		stringstyle=\color{colorUzLred}
	}     

	% \lstset{
	% 	backgroundcolor=\color{white},
	% 	basicstyle=\footnotesize\listingsfont,
	% 	breaklines=true, 
	% 	captionpos=b, 
	% 	commentstyle=\color{colorUzLgreen}\itshape,
	% 	escapeinside={\%*}{*)}, 
	% 	frame=single,
	% 	keepspaces=true,
	% 	keywordstyle=\color{colorUzLblue},
	% 	numbers=left,
	% 	numberstyle=\tiny\color{gray}, 
	% 	rulecolor=\color{black},  
	% 	showspaces=false, 
	% 	showstringspaces=false, 
	% 	showtabs=false, 
	% 	stepnumber=1, 
	% 	tabsize=2,title=\lstname,
	% 	language=C++,
	% 	stringstyle=\color{colorUzLred}
	% }

% 		\lstdefinelanguage{XMLSchema}
% 			{morekeywords={schema,element,annotation,appinfo,complexType,simpleType,choice,all,sequence},
% 			sensitive=true,
% %			morecomment=[l]{//},
% %			morecomment=[s]{/*}{*/},
% 			morestring=[b]",
% 		}

% 		\lstdefinelanguage{ASN1}
% 			{morekeywords={},
% 			sensitive=true,
% %			morecomment=[l]{//},
% %			morecomment=[s]{/*}{*/},
% 			morestring=[b]",
% 		}


	% Font
	%
	%	Danach muss man die Standardschriftart setzen mit dem Befehl \fontfamily{abr}\selectfont,
	% der für das gesamte restliche Dokument gilt, oder mit {\fontfamily{abr}\selectfont Some Text}
	% um nur den eingeklammerten Bereich zu betreffen. abr ist die Abkürzung für die Schriftart. Die
	% häufigsten sind ptm (Times), phv (Helvetica), pcr (Courier), pbk (Bookman), pag (Avant Garde),
	% ppl (Palatino), bch (Charter), pnc (New Century Schoolbook), pzc (Zapf Chancery), put (Utopia ).

	% Schönerer tt font:
		%\renewcommand{\ttdefault}{pcr}
		%\selectfont

	% Times
		%\usepackage{times}

	%	Helvetica
			%\usepackage{helvet}
			%\renewcommand{\familydefault}{\sfdefault}
			%\renewcommand{\familydefault}{phv}
			%\fontfamily{abr}\selectfont

	%	Courier
			%\usepackage{courier} \raggedright
			%\renewcommand{\familydefault}{\ttdefault}

	% Absatzformatierungen:
	% Keeps the distance between paragraphs constant
		\setlength{\parskip}{1.5ex plus 0.0ex minus 0.0ex}
		\setlength{\parindent}{0pt}

	% Zeilenabstand
		\renewcommand{\baselinestretch}{1.0}


	% % Chapter page must be set sparately
	% % http://tex.stackexchange.com/q/117328/8411
	% \fancypagestyle{plain}{%
	% 	\fancyhf{} % Delete all fields
	%     \renewcommand{\headrulewidth}{0pt}
	% 	\fancyfoot[EL,OR]{\thepage}
	% }

 %  % Generelles Page Style für die Hauptkapitel
	% \fancypagestyle{itm}{%
 %    \fancyhf{} % Delete all fields
 %    \renewcommand{\headrulewidth}{1pt}
 %    \fancyhead[EL]{\nouppercase{\leftmark}}
 %    \fancyhead[OR]{\nouppercase{\rightmark}}
 %    \fancyfoot[EL,OR]{\thepage}
	% }

 %  % Page style for Erklärung, Kurzfassung und Aufgabenstellung
	% \fancypagestyle{itmpreface}{%
 %    \fancyhf{} % Delete all fields
 %    \renewcommand{\headrulewidth}{0pt}
 %    \renewcommand{\footrulewidth}{0pt}
 %    \fancyfoot[EL,OR]{\thepage}
	% }

	% % Itemize look and feel
	% 	\renewcommand{\labelitemi}{\rule[+0.9mm]{2.7pt}{2.7pt}}
	% 	\renewcommand{\labelitemii}{--}
	% 	%\renewcommand{\labelitemiii}{}
	% 	%\renewcommand{\labelitemiv}{\#}


%------------------------------------------------------------------------------
%- Textbausteine
%------------------------------------------------------------------------------

	%Helpers
		\newcommand{\eigenname}[1]	{{\em #1}}

	%Deutsch
%		\newcommand{\figref}[1]{Abbildung~\ref{fig:#1}}
%		\newcommand{\tabref}[1]{Tabelle~\ref{tab:#1}}
%		\newcommand{\equref}[1]{Gleichung~\ref{equ:#1}}
%		\newcommand{\chapref}[1]{Kapitel~\ref{cha:#1}}
%		\newcommand{\appref}[1]{Anhang~\ref{cha:#1}}
%		\newcommand{\secref}[1]{Abschnitt~\ref{sec:#1}}
		\newcommand{\lstcref}[1]{Algorithmus~\cref{lst:#1}}
		\newcommand{\algref}[1]{Algorithmus~\ref{alg:#1}}
%		\newcommand{\ssecref}[1]{Unterabschnitt~\ref{ssec:#1}}
		\newcommand{\quoteref}[1]{Zitat~\ref{quote:#1}}


	%Englisch
		%\newcommand{\figref}[1]		{Figure~\ref{fig:#1}}
		%\newcommand{\tabref}[1]		{Table~\ref{tab:#1}}
		%\newcommand{\equref}[1]		{Equation~\ref{equ:#1}}
		%\newcommand{\algref}[1]		{Algorithm~\ref{alg:#1}}
		%\newcommand{\defref}[1]		{Definition~\ref{def:#1}}
		%\newcommand{\quoteref}[1]	{Quote~\ref{quote:#1}}

		%\newcommand{\chapref}[1]	{Chapter~\ref{cha:#1}}
		%\newcommand{\appref}[1]		{Appendix~\ref{cha:#1}}
		%\newcommand{\secref}[1]		{Section~\ref{sec:#1}}
		%\newcommand{\ssecref}[1]	{Section~\ref{ssec:#1}}
		%\newcommand{\sssecref}[1]	{Section~\ref{sssec:#1}}

	% REDEFINE UGLY STUFF
		\renewcommand{\leq}		{\leqslant}
		\renewcommand{\geq}		{\geqslant}
		\renewcommand{\epsilon}	{\varepsilon}
		\newcommand{\musec}		{$\mu sec$\xspace}
		\newcommand{\muW}		{$\mu W$\xspace}
		\newcommand{\plusminus}	{$\pm $\xspace}

%------------------------------------------------------------------------------
%- Worttrennung
%------------------------------------------------------------------------------

	%\hyphenation{Ge-samt-ozon}
	\hyphenation{name-space}
	\hyphenation{name-spaces}
	\hyphenation{ge-samten}

%------------------------------------------------------------------------------
%- Grafiken
%------------------------------------------------------------------------------

	\ifpdf
	  \DeclareGraphicsExtensions{.jpg,.pdf,.png}   % for pdftex driver
	\else
	  \DeclareGraphicsExtensions{.eps}             % for dvips driver
	\fi

	% Vereinfacht die Einbettung von Grafiken
	% Beispiel: \myfig[5cm]{psdatei}{Übersicht über das Gesamtsystem}
	\newcommand{\myfig}[3][\columnwidth]
	{
	 \begin{figure}[htbp]
		 \begin{center}
			 \includegraphics[width=#1]{#2}
			 \caption{#3}
			 \label{fig:#2}
		 \end{center}
	 \end{figure}
	}

	\newcommand{\myfigtwo}[4][\columnwidth]
	{
		 \begin{figure}[htbp]
				\begin{center}
				  \mbox
				  {
				    \subfigure[#2]
				    { \includegraphics[width=.45\columnwidth]{#1-a} \label{fig:#1-a} }
				    \quad
				    \subfigure[#3]
				    { \includegraphics[width=.45\columnwidth]{#1-b} \label{fig:#1-b} }
			    }
				  \caption{#4}
					\label{fig:#1}
				\end{center}
			\end{figure}
	}

	\newcommand{\myfigthree}[5][\columnwidth]
	{
		 \begin{figure}[htbp]
				\begin{center}
				  \mbox{
				    \subfigure[#2]
				    {
				    	\includegraphics[width=.3\columnwidth]{#1-a}
				    	\label{fig:#1-a}
				    }
				    \subfigure[#3]
				    {
							\includegraphics[width=.3\columnwidth]{#1-b}
				    	\label{fig:#1-b}
				    }
				    \subfigure[#4]
				    {
							\includegraphics[width=.3\columnwidth]{#1-c}
				    	\label{fig:#1-c}
				    }
			    }
				  \caption{#5}
					\label{fig:#1}
				\end{center}
			\end{figure}
	}

	\newcommand{\myfigfour}[6][\columnwidth]
	{
		 \begin{figure}[htbp]
				\begin{center}
				  \mbox
				  {
				    \subfigure[#2]
				    { \includegraphics[width=.45\columnwidth]{#1-a} \label{fig:#1-a} }
				    \quad
				    \subfigure[#3]
				    { \includegraphics[width=.45\columnwidth]{#1-b} \label{fig:#1-b} }
			    }
				  \mbox
				  {
				    \subfigure[#4]
				    { \includegraphics[width=.45\columnwidth]{#1-c} \label{fig:#1-c} }
				    \quad
				    \subfigure[#5]
				    { \includegraphics[width=.45\columnwidth]{#1-d} \label{fig:#1-d} }
			    }

				  \caption{#6}
				\label{fig:#1}
				\end{center}
			\end{figure}
	}

	% Unterverzeichniss(e) der einzubindenden Bilder,
	% wenn mehrere dann z.B. so: \graphicspath{{fotos/}{logos/}}
	\graphicspath{{images/}}

  % Deutscher Name für das Literaturverzeichnis
	\renewcommand\bibname{Bibliographie}



  \makeindex
